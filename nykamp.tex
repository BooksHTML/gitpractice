

\section{The boundary theta lift and linking numbers in Sol}

\subsection{Global theta functions for $W$}

We let $\calL_W$ be a $\G_P$-invariant (coset of a) lattice in $W$,
where $\G_N$ acts trivially on $W$. For $\varphi_{1,1}$, we define
its theta function by
\[
\theta_{\varphi_{1,1}}(\tau,{\calL_W})= \sum_{x \in \calL_W} \varphi_{1,1}(x,\tau)
\]
and similarly for $\psi_{0,1}$, and $\phi_{0,1}$. Then the usual
theta machinery gives that
$\theta_{\varphi_{1,1}}(\tau,{\calL_W})$ and $\theta_{\phi_{0,1}}(\calL_W)$
both transform like (non)-holomorphic modular forms of weight $2$
for some congruence subgroup of $\Sl_2(\Z)$.

\begin{remark}
The claim is not obvious for $\theta_{\phi_{0,1}}$, since $\phi_{0,1}$
is not a Schwartz function. In that case, we use
Proposition~\ref{phi-prop}. The component $B+B'$ of $\phi_{0,1}$
is $C^2$ outside the light cone. Since $W$ is anisotropic we can
then apply Possion summation, and this component transforms like a
modular form. Then apply the differential operator $X_{23}$ to
obtain the same for the other component $A+A'$ of $\phi_{0,1}$.
 
 In fact, if $W$ is isotropic and $\calL_W$ intersects non-trivially
 with the light cone, then $\theta_{\phi_{0,1}}$ is not quite a
 modular form. The case, when the $\Q$-rank of $V$ is $2$ is
 interesting in its own right. We will discuss this elsewhere.
\end{remark}






Via the map $\iota_P$ from Section~\ref{iotaP} we can view all theta
functions for $W$ as functions resp. differential forms on $e'(P)$.
We set $
\theta^P_{\varphi_{1,1}} =   \theta_{\varphi_{1,1}^P}$, 
and similarly $\theta^P_{\psi_{0,1}}$ and $\theta^P_{\phi_{0,1}}$.
Since $\iota_P$ is a map of complexes we immediately see by
Theorem~\ref{localholW} and Theorem~\ref{local-phi}

\begin{proposition}\label{globalholomorphic2}
\[
L \theta^P_{\varphi_{1,1}} = d\theta^P_{\psi_{0,1}} \qquad \text{and} \qquad 
L \theta^P_{\phi_{0,1}} = \theta^P_{\psi_{0,1}}.
\]
\end{proposition}

We now interpret the (holomorphic) Fourier coefficients of the
boundary theta lift associated to $ \theta^P_{\phi}(\tau,\calL_{W_P})$.
They are given by linking numbers. We have


\begin{theorem}\label{xi'-integralP}
Let $c$ a \bf{homological trivial} $1$-cycle in $e'(P)$ which
is disjoint from the torus fibers containing components of $\partial
C_n$ or for $c=\partial C_y$ for $C_y$ one of the components of
$C_n$, we have
\[
\int_{c} \theta^P_{\phi_{0,1}}(\tau,\calL_{W_P}) = 
  \sum_{n=1}^{\infty}  \Lk((\partial C_n)_P,c) q^n
 \; + \; \sum_{n \in \Q} \int_{c}  {\tilde{\psi}_{0,1}^P}(n)(\tau).
\]
 So the Fourier coefficients of the holomorphic part of $\int_{c} \theta^P_{\phi}(\tau,\calL_{W_P})$ are the linking numbers of the cycles $c$ and $ \partial C_n$ at the boundary component $e'(P)$.  
 \end{theorem}

Theorem \ref{xi'-integralP} follows from $\phi_{0,1} = \tilde{\psi}_{0,1} + \tilde{\psi'}_{0,1}$ combined with Theorem~\ref{linking-dual} below.  

\begin{example}\label{HZbeta}
In the situation of Examples~\ref{HZex} and \ref{LinkCnCmex}, we obtain
\[
\int_{\partial C_1} \theta^P_{\phi_{0,1}}(\tau,\calL_{W_P}) = \frac{1}{\sqrt{2d}} \sum_{\substack{\la \in \mathcal{O}_K \\ \la\la'>0}} \min(|\la|,|\la'|) e^{-2\pi \la \la' \tau } - \frac{\sqrt{2}}{\sqrt{dv}} \sum_{\la \in \mathcal{O}_K} \beta (\pi v (\la-\la')^2) e^{-2\pi \la \la' \tau},
\]
where $\beta(s) = \tfrac1{16\pi} \int_1^{\infty} e^{-st}t^{-3/2} dt$. This is (up to a constant) exactly Zagier's function $\mathcal{W}(\tau)$ in \cite{HZ}, \S 2.3. 


\end{example}

\subsection{Linking numbers, de Rham cohomology and linking duals} \label{generallinking}
We begin with a general discussion of integral formulas for linking numbers.
Such formulas go back to the classical Gauss-Amp\`ere formula for
$\R^3$, see \cite{F}, p.79-81, and \cite{DG} for its generalization
to $S^3$ and $H^3$.
Suppose now that $c$ is a $1$-cycle in an oriented compact $3$-manifold
M that is a rational boundary and $U$ is a tubular neighborhood of
$c$.
\begin{definition}
We will say any closed form $\beta$ in $M-U$ is a {\it linking dual}
(relative to $U$) of the bounding $1$-cycle $c$ if for any $1$-cycle
$a$ in $M-U$
which is a rational boundary in $M$ we have
\begin{equation*} \label{linkingdualdef} 
\int_a \beta = \Lk(a,c).
\end{equation*}
\end{definition}

