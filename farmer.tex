\section{The Hilbert modular surface and  its Borel-Serre compactification}


\subsection{The symmetric space and its arithmetic quotient}

\subsubsection{The orthogonal group and its symmetric space}


Let $V$ be a rational vector space of dimension $4$ with a
non-degenerate symmetric bilinear form  $(\,,\,)$ of signature
$(2,2)$.
We let $\underline{G} = \SO(V)$, viewed as an algebraic group over $\Q$.
We let $G=\underline{G}_0(\R) \simeq \SO_0(2,2)$ be the connected
component of the identity of the real points of $\underline{G}$.
It is most convenient to identify the associated symmetric space $D= D_V$ with the
space of negative $2$-planes in $V(\R)$ on which the bilinear form $(\,,\,)$ is
negative definite:
\[
D = \{z \subset V_{\R} ; \;\text{$\dim z =2$ and $(\,,\,)|_z < 0$}
\}.
\]
We pick an orthogonal basis $\{e_1,e_2,e_3,e_4\}$ of $V_{\R}$ with
$(e_1,e_1)=(e_2,e_2)=1$ and $(e_3,e_3)=(e_4,e_4)=-1$. We denote the
coordinates of a vector $x$ with respect to this basis by $x_i$.
We pick as base point of $D$ the plane $z_0=[e_3,e_4]$ spanned by
$e_3$ and $e_4$, and we let $K \simeq \SO(2)\times \SO(2)$ be the
maximal compact subgroup of $G$ stabilizing $z_0$. Thus $D \simeq
G/K$. Of course, $D \simeq \H \times \h$, the product of two upper
half planes.


We let  $\underline{P}$ be a rational parabolic subgroup stabilizing
a rational isotropic line $\ell$ and define $P= \underline{P}_0(\R)$
as before. We let $\underline{N}$ be its unipotent subgroup and $N
= \underline{N}(\R)$.
We let $u =(e_1+e_4)/\sqrt{2}$ and $u' =(e_1-e_4)/\sqrt{2}$ be two
isotropic vectors so that $(u,u')=1$. We assume that $u,u'$ are
defined over $\Q$ and obtain a rational Witt decomposition
\[
V = \ell \oplus W \oplus \ell'
\]
with $\ell = \Q u$, $\ell'=\Q u'$, and a subspace $W = \ell^{\perp}
\cap {\ell'}^{\perp}$ such that $W_{\R} = \Span_{\R}(e_2,e_3)$. The
choice of $u'$ gives a Levi splitting of $\underline{P}$, and we
write
\[
P =NAM
\]
for the Langlands decomposition. Here, with respect to the basis $u,e_2,e_3,u'$, we have  
\begin{align*}
{N} &=
 \{ n(w)   = ( 
\begin{smallmatrix}
1&(\cdot,w)& -(w,w)/2 \\
 &   1_W   & -w  \\
 &         & 1  \\
\end{smallmatrix} )
; \; w \in W_{\R} \}, \\ 
{A} &= \{ a(t) = 
( \begin{smallmatrix}
t& &  \\
 &   1_W   &  \\
 &         & t^{-1}\\
\end{smallmatrix} ); \, t \in \R_+
 \}, \\ 
{M} & =
\{ m(s)  =
( \begin{smallmatrix}
1& & \\
 &    \begin{smallmatrix} cosh(s) & sinh(s) \\ sinh(s) & cosh(s)
 \end{smallmatrix}    & \\\
 &         & 1  \\
\end{smallmatrix} )
; \; s \in \R \}.
\end{align*}
Note $N \simeq W_{\R}$. We obtain coordinates for $D$ by $z=z(t,s,w)$
where $z$ is the negative two-plane in $V_{\R}$ with
$z=[n(w)a(t)m(s)e_3,n(w)a(t)m(s)e_4]$.

