We also give a simple formula in Theorem~\ref{linkSol} for the linking number of two circles contained in the fiber of a $3$-manifold $M$ of type Sol in terms of the glueing homeomorphism for the bundle.

One can reformulate the previous theorem stating that $\sum_{n>0}\Lk(\partial C_n,c) q^n$ is a "mixed Mock modular form" of weight $2$; it is the product of a Mock modular form of weight $3/2$ with a unary theta series. Such forms, which originate with the famous Ramanujan Mock theta functions, have recently generated great interest. 

Theorem~\ref{FM-linking} (and its analogues for the Borel-Serre boundary of modular curves with non-trivial coefficients and Picard modular surfaces) suggest that there is a more general connection between modular forms and linking numbers of nilmanifold subbundles over special cycles in nilmanifold  bundles over locally symmetric spaces. 









\subsubsection*{Relation to the work of Hirzebruch and Zagier}

In their seminal paper \cite{HZ}, Hirzebruch-Zagier provided a map from the second homology of the smooth compactification of certain Hilbert modular surfaces $j:X \hookrightarrow \tilde{X}$ to modular forms. They introduced the Hirzebruch-Zagier curves $T_n$ in $X$, which are given by the closure of the cycles $C_n$ in $\tilde{X}$. They then defined "truncated" cycles $T_n^c$ as the
projections of $T_n$ orthogonal to the subspace of $H_2(\tilde{X},\Q)$ spanned by the 
compactifying divisors of $\tilde{X}$. The principal result of \cite{HZ} was that $\sum_{n \geq 0} [T_n^c] q^n$ defines a holomorphic modular form of weight $2$ with values in $H_2(\tilde{X},\Q)$.
We show $j_{\ast} C_n^c = T_n^c$ (Proposition~\ref{CnTn}), and hence the Hirzebruch-Zagier theorem follows easily from Theorem~\ref{FMHZ-main} above, see Theorem~\ref{HZTheorem}.

The main work in \cite{HZ} was to show that the generating function
\[\vspace{-.1cm}
F(\tau) = \sum_{n=0}^{\infty} (T_n^c \cdot T_m) q^n \vspace{-.1cm}
\]
for the intersection numbers in $\tilde{X}$ of $T^c_n$ with a fixed $T_m$ is a modular form of weight $2$. The Hirzebruch-Zagier proof of the modularity of $F$ was a remarkable synthesis of algebraic geometry, combinatorics, and modular forms. They explicitly computed the intersection numbers $T_n^c \cdot T_m$ as the sum of two terms, $T_n^c \cdot T_m = (T_n \cdot T_m)_X  + ({T}_n \cdot {T}_m)_{\infty}$, where $(T_n \cdot T_m)_X $ is the geometric intersection number of $T_n$ and $T_m$ in the interior of $X$ and $({T}_n \cdot {T}_m)_{\infty}$ which they called the "contribution from infinity". They then proved both generating functions $\sum_{n=0}^{\infty} (T_n \cdot T_m)_X  q^n$ and $\sum_{n=0}^{\infty}  (T_n \cdot T_m)_{\infty} q^n$ are the holomorphic parts of two non-holomorphic forms $F_X$ and $F_{\infty}$ with the {\it same} non-holomorphic part (with opposite signs). Hence combining these two forms gives $F(\tau)$.

