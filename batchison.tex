\section{Introduction}

In a series of papers \cite{FM1,FMcoeff,FMres,FMspec} we have been studying the geometric theta correspondence (see below) for non-compact arithmetic quotients of symmetric spaces associated to orthogonal groups. It is our overall goal to develop a general theory of geometric theta liftings in the context of the real differential geometry/topology of non-compact locally symmetric spaces of orthogonal and unitary groups which generalizes the theory of Kudla-Millson in the compact case, see \cite{KM90}.

In this paper we study in detail the geometric theta lift for Hilbert modular surfaces. In
particular, we will give a new proof and an extension (to all finite index subgroups of the
Hilbert modular group) of the celebrated theorem of Hirzebruch and Zagier \cite{HZ} that the generating function for the intersection numbers of the Hirzebruch-Zagier cycles is a classical modular form of weight $2$.\footnote{Eichler, \cite{HZ} p.104, proposed a proof using ``Siegel's work on indefinite theta functions''. This is what our proof is, though with perhaps more differential geometry than Eichler had in mind.} In our approach we replace Hirzebuch's smooth complex analytic compactification $\tilde{X}$ of the Hilbert modular surface $X$ with the (real) Borel-Serre compactification $\overline{X}$. The various algebro-geometric quantities that occur in \cite{HZ} are then replaced by topological quantities associated to $4$-manifolds with boundary. In particular, the ``boundary contribution'' in \cite{HZ} is replaced by sums of linking numbers of circles (the boundaries of the cycles) in the $3$-manifolds of type Sol (torus bundle over a circle) which comprise the Borel-Serre boundary.
