\subsubsection*{The cocycle of Kudla-Millson}

The key point of the work of Kudla and Millson \cite{KM1,KM2} is that they found (in greater generality) a family of cocycles $\varphi^V_{q}$ in $(\mathcal{S}(V) \otimes \wedge^q \mathfrak{p}^{\ast})^K$ with weight $(p+q)/2$ for $\SL_2$. Moreover, these cocycles give rise to Poincar\'e dual forms for certain totally geodesic, ``special'' cycles in $X$. Recently, it has now been shown, first \cite{HoffmanHe} for $\SO(3,2)$, and then \cite{BMM} for all $\SO(p,q)$ and $p+q>6$ (with $p \geq q$) in the cocompact (standard arithmetic) case that the geometric theta correspondence specialized to $\varphi_q^V$ induces on the adelic level an {\it isomorphism} from the appropriate space of classical modular forms to $H^q(X)$. In particular, for any congruence quotient, the dual homology groups are spanned by special cycles. This gives further justification to the term geometric theta correspondence and highlights the significance of these cocycles. In \cite{FMcoeff} we generalize $\varphi^V_{q}$ to allow suitable non-trivial coefficient systems (and one has an analogous isomorphism in \cite{BMM}). 
\\[12pt] 
\textbf{2.3.4 The main results} 
\\[10pt]
In the present paper, we consider the case when $V$ has signature $(2,2)$ with $\Q$-rank $1$. Then $D \simeq \h \times \h$, and $X$ is a Hilbert modular surface. We let $\overline{X}$ be the Borel-Serre compactification of $X$ which is obtained by replacing each isolated cusp associated to a rational parabolic $P$ with a boundary face $e'(P)$ which turns out to be a torus bundle over a circle, a $3$-manifold of type Sol. This makes $\overline{X}$ a $4$-manifold with boundary.  For simplicity, we assume that $X$ has only one cusp so that $\partial \overline{X} = e'(P)$, and we write $k: \partial \overline{X}  \hookrightarrow \overline{X}$ for the inclusion. The special cycles $C_n$\footnote{We distinguish the relative cycles $C_n$ in $X$ from the Hirzebruch-Zagier cycles $T_n$ in $\tilde{X}$, see below.} in question are now embedded modular and Shimura curves, and are parameterized by $n \in \N$. They define relative homology classes in $H_2(X, \partial X,\Q)$. 