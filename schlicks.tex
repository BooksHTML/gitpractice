\subsection{Proof of Theorem~\ref{linking-dual}}\label{W-currents}

We now prove Theorem~\ref{linking-dual}. First we will assume that we are in case(i). We need to show  
\[
\int_c  \tilde{\psi'}_{0,1}(n) = \Lk(\partial C_n,c) e^{-2\pi n} =  (A_n \cdot c) e^{-2\pi n}.
\]
The theorem will be a consequence of the following discussion.  
We may assume that $c$ is an embedded loop in $e'(P)$ (note that
since any loop in a manifold of dimension $3$ or more is homotopic
to an embedded loop by transversality any homology class of degree
$1$ in $e'(P)$ is represented by an embedded loop).

Choose a tubular neighborhood $N(c)$ of $c$ such that $N(c)$ is
disjoint from $F_n$. Let $\eta_c$ be a closed $2$-form which is
supported inside $N(c)$ and has integral $1$ on the disk fibers of
$N(c)$ (a Thom class for the normal disk bundle $N(c)$).
Then we have proved in Subsection \ref{generallinking}  

\begin{lemma}\label{integrallink}
\begin{equation} \label{secondformula}
\int_{A_n} \eta_c= A_n \cdot  c= \Lk(\partial C_n, c).
\end{equation}
\end{lemma}


We then have

\begin{lemma} \label{laststep}
\begin{equation*} \label{fifthformula}
\int_c \tilde{\psi'}_{0,1}(n) = \left(\int_{A_n} \eta_c\right) e^{- 2\pi n}.
\end{equation*}
\end{lemma}

\begin{proof}
To prove the Lemma we compute $\int_{e'(P)} \eta_c \wedge  \tilde{\psi'}_{0,1}(n)= \int_{e'(P)} \tilde{\psi'}_{0,1}(n) \wedge \eta_c$ in two different ways. First we apply Proposition \ref{finalintegral} with $\eta = \eta_c$.  We deduce
\begin{equation*}\label{thirdformula}
\int_{e'(P)}\eta_c \wedge \tilde{\psi'}_{0,1}(n) = \left(\int_{A_n} \eta_c\right) e^{-2 \pi n}.
\end{equation*}
Next choose a tubular neighborhood $V_n$ of the fibers $F_n$   such that $e'(P) - V_n$
contains $N(c)$.  Then $\tilde{\psi'}_{0,1}(n)$ is smooth on $e'(P)
- V_n \supset \supp (\eta_c)$.  Also, since $\eta_c$ is the extension
of a Thom class by zero,  the restriction of  $\eta_c$ to $e'(P) -
V_n$ represents the Poincar\'e dual $PD(c)$ of the absolute cycle
$c$ in
$e'(P) -V_n$. The lemma now follows from 
\begin{equation*}
\int_{e'(P)} \tilde{\psi'}_{0,1}(n) \wedge \eta_c   = \int_{e'(P)- V_n} \tilde{\psi'}_{0,1}(n) \wedge \eta_c = \int_{e'(P)- V_n} \tilde{\psi'}_{0,1}(n) \wedge PD(c)
  = \int_{c} \tilde{\psi'}_{0,1}(n). 
\end{equation*}
\end{proof}

By Lemma \ref{integrallink} this concludes the proof of Theorem~\ref{linking-dual} in the case when $c$ is disjoint from the fibers $F_n$.


It remains to treat case (ii). 
Thus we now assume that $c=c_y$ which is contained in a fiber $F_x$ containing a component  of $\partial C_n$.  We first prove
\begin{lemma}\label{selflinkingforx}
\[
\int_{c} \widetilde{\psi}'_{0,1}(x) =
\int_{c(\epsilon)}  \widetilde{\psi}'_{0,1}(x).
\]
\end{lemma}
\begin{proof} 
We can take $x = \mu e_2$ and hence $c$ is contained in the fiber over the image of $e_3 \in W$. Hence, by proposition \ref{boundaryofC}, $c $ is the circle in 
the torus fiber at $s(x )=0$ {\it in the $e_3$-direction}, 
i.e., parallel to the image of $(0,\R e_3)$ in $e'(P)$. 
We note that by \eqref{psi'formula} even though 
$\widetilde{\psi}'_{0,1}(x)$ is not defined on the 
whole fiber over $s=0$ its restriction to $c$ is smooth. 
Hence the left hand side is well-defined since all the 
other terms in the sum are defined on the whole fiber 
and in fact in a neighborhood of that fiber. Hence the 
locally constant form $\sum_{\gamma \in \Gamma_M} \gamma^* \widetilde{\psi}'_{0,1}(x)$ 
is closed on the cylinder $[0, \epsilon] \times c $, 
and its integrals over the circles $s \times c $ all 
coincide. But $ \eps \times c = c(\epsilon)$. The lemma follows. 
\end{proof} 
Summing over $x$ and using case (i) we obtain
\[
\int_{c} \widetilde{\psi}'_{0,1}(n) =\int_{c(\epsilon)}  \widetilde{\psi}'_{0,1}(n) = \Lk(\partial C_n ,c(\epsilon)),
\]
since $c(\epsilon)$ is disjoint from all the components of $F_n$. Thus it suffices to prove 
\begin{equation}\label{lastlinkingstep}
\Lk(\partial C_n, c) = \Lk(\partial C_n, c(\epsilon)).
\end{equation}
To this end suppose that $c \subset F_x \subset F_n$ and $c_1,\cdots,c_k$ are the components of $\partial C_n$ contained in $F_x$. Hence
$c$ and $c_i,1 \leq i \leq k$, are all parallel. Since the fibers containing all other components of $\partial C_n$ are disjoint from $c$, \eqref{lastlinkingstep} will follow from
\[
\Lk(c_i,c) = \Lk(c_i, c(\epsilon)), 1 \leq i \leq k.
\]
If $c_i = c$ then the previous equation is the definition of $Lk(c,c)$. Thus we may assume $c_i$ is parallel to and disjoint from $c$. 
In this case their linking number is already topologically defined.
But since $c$ is disjoint from $c_i$ the circles $c$ and $c(\eps)$
are homologous in the complement of $c_i$ (by the product homology
$c \times [0,\eps]$) and since the linking number with $c_i$ is a
homological invariant of the complement of $c_i$ in $e'(P)$ we have
$\Lk(c_i, c(\eps)) =\ Lk(c_i, c)$.

We see that this Theorem~\ref{linking-dual} is proved. 




