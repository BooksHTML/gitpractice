Let $h_1$ resp.\ $h_2$ denote the covering transformation of $\pi$
corresponding to the element $\alpha_0$ resp $\gamma_0$ of the
fundamental group of $T^2$. Define $c_1$ and $c_2$ in $W$ by $c_1
= Nh_1(0)$ and $c_2=h_2(0)$. Define $d \in W$ by $d =f^{-1}(c_2)$
in $W$. Let $\widetilde{T}$ be the oriented triangle
with vertices $0,c_2,d$.  Then
\begin{equation*}
 (i) \quad \pi(\overline{0c_1}) =N\alpha_0, \qquad
 (ii) \quad \pi(\overline{0c_2})= \gamma_0, \qquad
 (iii) \quad \pi(\overline{0d}) = \pi(f^{-1}(\overline{0c_2})) = f^{-1}(\gamma_0).
\end{equation*}
We now leave it to the reader to combine the homology equation
\eqref{invertmatrix} and the three equations to show the equality
of directed line segments
\begin{equation}\label{equalityofsegments}
h_2( \overline{0c_1}) = \overline{c_2d}.
\end{equation}
With this we can solve the problem. 
We see that if we consider $\widetilde{T}$ as an oriented $2$-simplex
we have the following equality of one chains
\begin{equation*}
\partial \widetilde{T} = \overline{0c_2} + \overline{c_2d} - \overline{0d}.
\end{equation*}
Let $T$ be the image of $\widetilde{T}$ under $\pi$. Take
the direct image of the previous equation under $\pi$ and use
equation \eqref{equalityofsegments}, which implies that the second
edge $\overline{c_2d}$ is equivalent under $h_2$ in the covering
group to the directed line segment $\overline{0c_1}$, which maps to
$N\alpha_0$. Hence $\overline{c_2d}$ also maps to $N\alpha_0$. We
obtain  the following equation in $Z_1(T^2,\Z)$
\begin{equation}\label{triangle}
\partial T =  \gamma_0 + N \alpha_0 - f^{-1}(\gamma_0),
\end{equation}
and we have solved the above problem. Combining \eqref{boundaryofmon}
and \eqref{triangle} we have
\begin{equation*}
\partial (\mathcal{M}(\gamma_0) + T ) = f^{-1}(\gamma_0) -\gamma_0 +\gamma_0 + \alpha_0 - f^{-1}(\gamma_0)= N\alpha_0.
\end{equation*}
Combining this with \eqref{firstrectangle} and setting where $A_0 =
\mathcal{M}(\gamma_0) +T$, we obtain
\begin{equation} \label{cap}
\partial (NP + A_0 ) = N \alpha,
\end{equation}
in $Z_1(M,\Z)$. Hence if we define $A$ to be the  \emph{rational}
chain $A = \frac{1}{N} (NP + A_0) = P + \frac{1}{N}T + \frac{1}{N}
\mathcal{M}(\gamma_0)$ in $M$ we have the following equation in
$Z_1(M,\Q)$:
\begin{equation*}
\partial A = \alpha.
\end{equation*}
Finally, the integral of $\Omega$ over $A$ is rational. Indeed, the
integral over $P$ is rational.  Since all vertices of $\widetilde{T}$
are integral the area of $\widetilde{T}$ is integral, the integral
of $\Omega$ over $T$ is integral. Thus it suffices to observe that
the restriction of $\Omega$ to $\mathcal{M}(c)$ is zero. With this
we have completed the proof of Proposition \ref{rat-cap}.

