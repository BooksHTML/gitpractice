The geometric theta correspondence of Kudla-Millson \cite{KM90} for
the cocycle $\varphi^V_{2}$ in this situation takes the following
shape. For a {\it compact} cycle $C$ in $X$, we have that
\begin{equation}\label{KM-id}
\langle \theta_{\varphi^V_{2}}, C \rangle =  \int_C  \theta_{\varphi^V_{2}}= \sum_{n \geq 0} (C_n \cdot C) q^n
\end{equation}
is a holomorphic modular form of weight $2$ and is equal to the
generating series of the  intersection numbers with $C_n$. Here $q
= e^{2\pi i \tau}$ with $\tau \in \h$.
(There is a similar statement for the pairing of $\theta_{\varphi^V_{2}}$
with a closed {\it compactly supported} differential $2$-form on
${X}$ representing a class in $H^2_c(X)$, see Theorem~\ref{KM90}).
Our first result is

\begin{theorem}\label{FM-boundaryexact}  (Theorem~\ref{globalexact})
The {\bf differential form} $\theta_{\varphi^V_{2}}$ on $X$ extends
to a form on $\overline{X}$, and the restriction $k^{\ast}$ of
$\theta_{\varphi^V_{2}}$ to $\partial \overline{X}$ gives an {\it
exact} differential form on $\partial\overline{X}$. Moreover, there
exists a theta series $\theta_{\phi_1^W}$ for a space $W$ of signature
$(1,1)$  of weight $2$ with values in the $1$-forms on $\partial
\overline{X}$ such that $\theta_{\phi_1^W}$ is a primitive for
$k^{\ast} \theta_{\varphi^V_{2}}$:
\[
d (\theta_{\phi_1^W}) = k^{\ast} \theta_{\varphi^V_{2}}.
\]
\end{theorem}

Considering the mapping cone for the inclusion $k: \partial
\overline{X} \hookrightarrow \overline{X}$ (see
Section~\ref{mappingconesection}) we then view the pair
$[\theta_{\varphi^V_{2}}, \theta_{\phi_1^W}]$ as an element of the
compactly supported cohomology $H^2_c(X)$. Explicitly, let $C$ be a
relative cycle in $\overline{X}$ representing a class in $H_2({X},\partial
{X},\Z)$. Then the Kronecker pairing between $[\theta_{\varphi^V_{2}},
\theta_{\phi_1^W}]$ and $C$ is given by
\begin{equation}\label{mappingconelift}
\langle [\theta_{\varphi^V_{2}}, \theta_{\phi_1^W}], C \rangle = \int_C  \theta_{\varphi^V_{2}}
 - \int_{\partial C} \theta_{\phi_1^W}.
\end{equation}
In this way, we obtain an extension of the geometric theta lift
which captures the non-compact situation.

To describe the geometric interpretation of this extension, we study
the cycle $C_n$ at the boundary $\partial \overline{X}$
(Section~\ref{capped-cycles}). The intersection of $C_n$ with
$\partial \overline{X}$ is a union of circles contained in the torus
fibers of Sol. But rationally such circles are homologically trivial.
Hence we can find a (suitably normalized) rational $2$-chain $A_n$
in $\partial \overline{X}$ whose boundary is the boundary of $C_n$
in $\partial \overline{X}$. ``Capping'' off $C_n$ by $A_n$, we
obtain a {\it closed} cycle $C_n^c$ in $\overline{X}$ defining a
class in $H_2({X},\Q)$. Our main result is the extension of
\eqref{KM-id}:
\begin{theorem}\label{FMHZ-main} (Theorem~\ref{FM-main-th})
 Let $C$ be a relative cycle in $\overline{X}$. Then
\[
\langle [\theta_{\varphi^V_{2}}, \theta_{\phi_1^W}], C \rangle  = \sum_{n \geq 0} (C_n^c \cdot C)q^n
\]
is a holomorphic modular form of weight $2$ and is equal to the
generating series of the  intersection numbers with the capped
cycles $C^c_n$. (Similarly for the pairing with an arbitrary closed
$2$-form on $\overline{X}$ representing a class in $H^2(X)$).
\end{theorem}
