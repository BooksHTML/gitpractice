\section{Schwartz functions and forms}

Let $U$ be a non-degenerate rational quadratic space of  signature $(p,q)$ and even dimension $m$. We will later apply the following to $U=V$ and $U=W$. Changing notation from before, we let $G = \SO_0(U_{\R})$ with maximal compact subgroup $K$ and write $D=G/K$ for the associated symmetric space. We let $\calS(U_{\R})$ be the space of Schwartz functions on $U_{\R}$ on which $\SL_2(\R)$ acts via the Weil representation $\omega$.

\subsection{Extending certain Schwartz functions to functions of $\tau \in \h$ and $z\in D$}\label{conventions}

Let $\varphi \in \calS(U_{\R})$ be an eigenfunction under the maximal compact $\SO(2)$ of $\SL_2(\R)$ of
weight $r$. Define $g'_{\tau} \in \SL_2(\R)$
by $g'_{\tau} = \left(
\begin{smallmatrix}1&u\\0&1\end{smallmatrix} \right) \left(
\begin{smallmatrix}v^{1/2}&0\\0&v^{-1/2}\end{smallmatrix} \right)$.
Then we have $ \omega(g'_{\tau}) \varphi (x)= v^{m/4} \varphi(\sqrt{v}x) e^{\pi i (x,x)u}$.
Accordingly we define 
\begin{equation}\label{group-tau}
\varphi(x,\tau)  = v^{-r/2} \omega(g'_{\tau}) \varphi
(x)  = v^{-r/2+m/4} \varphi^0(\sqrt{v}x) e^{\pi i (x,x)\tau}.
\end{equation}
Here we have also defined $\varphi^0(x) = \varphi(x) e^{\pi (x,x)}$. Let $E$ be a $G$-module and let $g_z \in G$ be any element that carries the basepoint $z_0$ in $D$ to $z \in D$. Then define for $\varphi \in [\calS(U_{\R}) \otimes E]^K$, the $E$-valued $K$-invariant Schwartz functions on $U_{\R}$, the functions $\varphi(x,z)$ and $\varphi(x,\tau,z)$ for $x \in U, z \in D, \tau \in \mathbb{H}$ by
\[
\varphi(x,z) =g_z \varphi(g_z^{-1}x) \qquad  \text{and} \qquad  \varphi(x,\tau,z) = g_z\varphi(g_z^{-1}x,\tau).
\]
We will continue to use these notational conventions for other (not necessarily Schwartz) functions that arise in this paper.


\subsection{Schwartz forms for $V$}\label{V-forms}

Let $\mathfrak{g} $ be the Lie algebra of $G$ and $\mathfrak{g}= \mathfrak{k} \oplus \mathfrak{p}$ be the Cartan decomposition of $\mathfrak{g}$ associated to $K$.  We identify 
 $\mathfrak{g} \simeq \wwedge{2} V_{\R}$ as usual via $
(v_1 \wedge v_2)(v) = (v_1,v)v_2 - (v_2,v)v_1$. We write $X_{ij} = e_i \wedge e_j \in \mathfrak{g}$ and note that $\mathfrak{p}$ is spanned by $X_{ij}$ with $1 \leq i \leq 2$ and $3 \leq j \leq 4$. We write $\omega_{ij}$ for their dual. We orient $D$ such that $\omega_{13} \wedge \omega_{14} \wedge \omega_{23} \wedge \omega_{24}$ gives rise to the $G$-invariant volume element on $D$. 
\\[12pt] 
\textbf{5.2.1 Special forms for $V$} 
\\[10pt]
IThe Kudla-Millson form $\varphi_2$ is an element in 
\[
 [\calS(V_{\R}) \otimes \calA^2(D)]^G\simeq
[\calS(V_{\R}) \otimes \wwedge{2} \mathfrak{p}^{\ast}]^K,
\]
where the isomorphism is given by evaluation at the base point. Here $\calA^2(D)$ denotes the differential $2$-forms on $D$. Note that $G$ acts diagonally in the natural fashion. At the base point $\varphi_2$ is given by
\[
\varphi_2= \frac12 \prod_{\mu=3}^4 \sum_{\alpha=1}^{2}  \left( x_{\alpha} - \frac1{2\pi}\frac{\partial}{\partial x_{\alpha}} \right) \varphi_0 \otimes \omega_{\alpha\mu}.
\]
Here $\varphi_0(x) := e^{-\pi(x,x)_{0}}$, where $(x,x)_0= \sum_{i=1}^4 x_i^2$ is the minimal majorant associated to the base point in $D$. Note that $\varphi_2$ has weight $2$, see \cite{KM1}. There is another Schwartz form $\psi_1$ of weight $0$ which lies in $
[\calS(V_{\R}) \otimes \calA^1(D)]^G\simeq
[\calS(V_{\R}) \otimes \mathfrak{p}^{\ast}]^K$ and is given by
\begin{equation}\label{psi20}
\psi_1 =  -x_1x_3\varphi_0(x) \otimes \omega_{14}+x_1x_4  \varphi_0(x) \otimes \omega_{13} - x_2x_3 \varphi_0(x) \otimes \omega_{24}+x_2x_4\varphi_0(x) \otimes \omega_{23}. 
\end{equation}


The key relationship is (see \cite{KM90}, \S 8)

\begin{theorem}\label{localholomorphic1}
\[
\omega(L) \varphi_2 = d \psi_1.
\]
Here $\omega(L)$ is the Weil representation action
of the $\SL_2$-lowering operator $L = \tfrac12 \left(
\begin{smallmatrix}1 & -i \\ -i & -1\end{smallmatrix} \right)  \in \mathfrak{sl}_2(\R)$ on
$\calS(V_{\R})$, while $d$ denotes the exterior differentiation on $D$. 
\end{theorem}

On the upper half plane $\h$, the action of $L$ corresponds to the action of the classical Maass lowering operator which we also denote by $L$. For a function $f$ on $\h$, we have
\[
Lf  = -2iv^2 \frac{\partial}{\partial \bar{\tau}} f.
\]
When made explicit using \eqref{group-tau} Theorem \ref{localholomorphic1} translates to
\begin{equation}\label{partial-d}
v \frac{\partial}{\partial v }  \varphi_2^0(\sqrt{v}x) =  d
\left(\psi_1^0(\sqrt{v}x)\right).
\end{equation}

